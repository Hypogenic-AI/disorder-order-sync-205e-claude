\section{Discussion}\label{sec:discussion}

\subsection{Connections to prior work}

Our results relate to three lines of research: the barycentric
condition framework of Palacios, In, and Amani \cite{Palacios2024},
the asymmetry-induced synchronization (AISync) theory of Zhang,
Nishikawa, and Motter \cite{Zhang2017}, and the feedforward network
analysis of Ahmed et al.\ \cite{Ahmed2026}.

\medskip\noindent\textbf{Barycentric condition.}\quad
Proposition~\ref{prop:barycentric-necessary} confirms the central
thesis of \cite{Palacios2024}: the barycentric condition is a
necessary constraint when measuring the effect of heterogeneity on
synchronization.  We extend this by showing that the condition is far
from sufficient---among the six topologies tested, only ring networks
admit disorder-enhanced synchronization in the Kuramoto framework, and
even there the enhancement is realization-dependent
(Proposition~\ref{prop:realization}).

\medskip\noindent\textbf{AISync and converse symmetry breaking.}\quad
The circulant graph survey (Theorem~\ref{thm:circulant-fraction})
yields fractions ($20\%$--$37\%$) consistent with the $10\%$--$50\%$
range for AISync prevalence reported in \cite{Zhang2017}.  However,
our setup differs in important ways: we use Kuramoto (phase-only)
oscillators with frequency disorder, whereas Zhang et al.\ use general
oscillators with structural heterogeneity via a multilayer
construction.  The close agreement suggests that the AISync phenomenon
is more universal than the specific multilayer mechanism used to
construct it.

The experimental demonstration of converse symmetry breaking by
Molnar, Nishikawa, and Motter \cite{Molnar2020} used
electromechanical oscillators (swing equation model) on a three-node
rotationally symmetric network.  Our results extend this to larger
networks ($N = 6$--$20$) and a different oscillator model (Kuramoto),
finding qualitatively similar behavior: disorder helps, but only for
specific realizations on suitable topologies.

\medskip\noindent\textbf{Feedforward networks.}\quad
Theorem~\ref{thm:SL-enhancement} provides independent computational
verification of the analytical predictions of Ahmed et al.\
\cite{Ahmed2026} and extends them in two directions.  First, we
quantify the monotonicity of the phase-locking fraction as a function
of disorder strength (Theorem~\ref{thm:SL-enhancement}(b)), which was
not explicitly computed in \cite{Ahmed2026}.  Second,
Proposition~\ref{prop:three-cell} extends the analysis to three-cell
chains and identifies the directionality principle: excitation
increasing along the feedforward chain enhances output.

\subsection{Spectral mechanisms}

The spectral gap ratio $\rho(G)$ emerges as a key predictor of whether
a topology can benefit from disorder
(Observation~\ref{obs:spectral-predictor}).  The mechanism operates
through the master stability function framework
(Definition~\ref{def:msf} and Remark~\ref{rem:msf}): when the MSF
stability region is a bounded interval $(\alpha_1, \alpha_2)$,
synchronization requires $\rho(G) < \alpha_2 / \alpha_1$.  Graphs
with large $\rho$ are close to this boundary, making them sensitive to
perturbations of the effective Laplacian.

For degree-regular graphs (such as rings and circulant graphs), a
specific disorder realization $\omega^*$ induces phase offsets
$\phi_i$ that modify the effective Laplacian as $L_{\mathrm{eff}} = L
\circ \cos\Phi$, where $\circ$ denotes the Hadamard product and
$(\cos\Phi)_{ij} = \cos(\phi_j - \phi_i)$.  If the disorder
realization is ``aligned'' with the Fiedler eigenvector of $L$ in a
suitable sense, it can selectively reduce $\lambda_N(L_{\mathrm{eff}})$
while approximately preserving $\lambda_2(L_{\mathrm{eff}})$,
decreasing $\rho(L_{\mathrm{eff}})$ and improving synchronizability.

This explanation is consistent with two observations.  First, the
enhancement is realization-dependent
(Proposition~\ref{prop:realization}): random realizations are equally
likely to increase or decrease $\rho(L_{\mathrm{eff}})$, so the
average effect is null.  Second, the optimal disorder strength is
small (Lemma~\ref{lem:small-disorder}): the perturbation to
$L_{\mathrm{eff}}$ is second-order in the disorder, so a small
$\delta$ suffices to shift eigenvalues, while large $\delta$ disrupts
the perturbative regime.

\subsection{Amplitude--stability trade-off}

A recurring theme in our results is the trade-off between amplitude
(signal strength) and stability (robustness of phase locking).  In
Stuart--Landau feedforward networks, increasing disorder monotonically
improves phase locking while monotonically decreasing peak amplitude
(Theorem~\ref{thm:SL-enhancement}(b)--(c)).  This trade-off is
practically relevant: in applications such as signal amplification in
biological or engineered networks, one must choose between fidelity
(large amplitude) and reliability (broad phase locking).

The trade-off can be understood via the energy budget.  In the
heterogeneous case, the output cell is below its Hopf bifurcation
($\mu_2 < 0$) and dissipates energy from the input signal rather than
contributing its own autonomous oscillation.  This dissipation reduces
the peak amplitude but eliminates the frequency competition between
the input and output cells that can destabilize phase locking.

\subsection{Open questions and conjectures}

Our results suggest several directions for further investigation.

\begin{conjecture}[Spectral criterion for disorder enhancement]
\label{conj:spectral}
A connected, degree-regular graph $G$ on $N$ vertices admits a
zero-mean frequency realization $\omega^*$ that enhances Kuramoto
synchronization (in the sense of Theorem~\ref{thm:ring-enhancement})
if and only if $\rho(G) > c \cdot N$ for some universal constant
$c > 0$.
\end{conjecture}

The evidence for this conjecture is limited to the six topologies in
our study.  The ring $C_N(1)$ has $\rho \approx 4 / (2\pi/N)^2 \sim
N^2$, which far exceeds any linear threshold.  A proof would likely
require tight bounds on how zero-mean perturbations to phase
oscillators modify the effective Laplacian spectrum.

\begin{conjecture}[Optimal disorder via Fiedler eigenvector]
\label{conj:fiedler}
For a degree-regular graph $G$ with Fiedler eigenvector $v_2$, the
optimal zero-mean disorder realization for enhancing synchronization
is proportional to $v_2$, i.e., $\omega_i^* \propto (v_2)_i$ subject
to $\sum \omega_i^* = 0$.
\end{conjecture}

This conjecture is motivated by the observation that the Fiedler
eigenvector identifies the ``weakest direction'' of the synchronization
manifold.  Disorder aligned with this direction could selectively
reduce the effective $\lambda_N$ without reducing $\lambda_2$, as the
Fiedler mode is already saturated.  Verifying this conjecture
would require a careful perturbation analysis of the effective
Laplacian eigenvalues.

\begin{conjecture}[Thermodynamic limit]\label{conj:thermo}
For the Kuramoto model on the ring graph $C_N(1)$ in the limit
$N \to \infty$, the set of zero-mean frequency distributions $g$ for
which disorder enhances synchronization has measure zero in the space
of distributions.  That is, the realization dependence observed in
Proposition~\ref{prop:realization} becomes a distributional
impossibility in the thermodynamic limit.
\end{conjecture}

If true, this conjecture would explain the classical intuition that
heterogeneity hurts synchronization: for infinite populations, the
self-averaging of random frequencies eliminates any beneficial
correlations.  The finite-$N$ results of
Theorem~\ref{thm:ring-enhancement} would then represent a genuine
finite-size effect that vanishes as $N \to \infty$.
