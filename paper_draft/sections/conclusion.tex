\section{Conclusion}\label{sec:conclusion}

We have conducted a systematic numerical investigation of
disorder-enhanced synchronization in coupled oscillator networks,
testing when zero-mean parameter heterogeneity satisfying the
barycentric condition \cite{Palacios2024} can broaden or stabilize
synchronization regions.  Our main contributions are:

\begin{enumerate}[label=(\arabic*)]
\item \textbf{Stuart--Landau feedforward networks} provide the
  strongest evidence for disorder-enhanced synchronization:
  excitation-parameter disorder nearly doubles the phase-locking
  region (Theorem~\ref{thm:SL-enhancement}), with a monotonic
  increase in phase-locking fraction from $61\%$ to $100\%$.  The
  direction of excitation disorder along the feedforward chain
  determines whether it enhances or suppresses output
  (Proposition~\ref{prop:three-cell}).

\item \textbf{Kuramoto ring networks} exhibit realization-dependent
  disorder enhancement: specific zero-mean frequency patterns improve
  the order parameter by up to $32\%$
  (Theorem~\ref{thm:ring-enhancement}), but averaging over random
  realizations yields a nonsignificant or negative effect
  (Proposition~\ref{prop:realization}).

\item \textbf{The barycentric condition is necessary but not
  sufficient} for disorder to help
  (Proposition~\ref{prop:barycentric-necessary}).  The spectral gap
  ratio $\rho(G)$ of the graph Laplacian emerges as a predictor of
  which topologies can benefit
  (Observation~\ref{obs:spectral-predictor}), and the optimal disorder
  strength is small (Lemma~\ref{lem:small-disorder}).

\item \textbf{The fraction of circulant graphs} benefiting from
  disorder ranges from $20\%$ to $37\%$ for $N \in \{6, 8, 10\}$
  (Theorem~\ref{thm:circulant-fraction}), consistent with the AISync
  predictions of \cite{Zhang2017}.
\end{enumerate}

These results demonstrate that disorder-enhanced synchronization is a
genuine but highly structured phenomenon.  Beneficial disorder is not
random: it must be aligned with the network topology in specific ways.
This structural requirement limits the practical applicability of
``disorder as a design tool,'' but also opens the possibility of
deliberately engineering heterogeneity patterns for targeted
synchronization enhancement---a direction that warrants both further
theoretical analysis (Conjectures~\ref{conj:spectral}--\ref{conj:thermo})
and experimental validation with physical oscillator networks.
