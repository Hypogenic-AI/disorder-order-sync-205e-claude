\section{Introduction}\label{sec:intro}

Synchronization of coupled oscillators is a ubiquitous phenomenon in
physics, biology, and engineering, from the flashing of fireflies to
the stability of electrical power grids
\cite{Strogatz2000,Pikovsky2001,Acebron2005}.  Since
Kuramoto's seminal mean-field model \cite{Kuramoto1984}, it has been
understood that parameter heterogeneity---differences in the natural
frequencies of individual oscillators---generically \emph{hinders}
synchronization.  More precisely, for globally coupled phase
oscillators with natural frequencies drawn from a symmetric unimodal
distribution $g(\omega)$, the critical coupling strength satisfies
$K_c = 2/(\pi g(0))$ \cite{Kuramoto1984,Strogatz2000}, which
increases as the distribution broadens.  For network-coupled
oscillators, D\"orfler and Bullo \cite{Dorfler2014} showed that the
critical coupling scales as $\lVert\omega\rVert_{\mathcal{E},\infty}
/ \lambda_2(L)$, where $\lambda_2(L)$ is the algebraic connectivity
of the network Laplacian: wider frequency spread requires stronger
coupling.

Recent theoretical developments have challenged this classical
picture.  Zhang, Nishikawa, and Motter \cite{Zhang2017} introduced
\emph{asymmetry-induced synchronization} (AISync), demonstrating that
for certain symmetric network topologies, stable synchronization is
\emph{impossible} with identical oscillators but becomes possible when
oscillator parameters are made heterogeneous.  This phenomenon was
experimentally confirmed by Molnar, Nishikawa, and Motter
\cite{Molnar2020} using coupled electromechanical oscillators.
Independently, Palacios, In, and Amani \cite{Palacios2024} used
normal-form analysis to study disorder-induced dynamics in complex
networks, establishing the \emph{barycentric condition}: when studying
the effect of heterogeneity on synchronization, the mean of the
parameter distribution must coincide with the bifurcation point.
Failure to impose this condition can lead to spurious conclusions about
disorder enhancing synchronization.  Most recently, Ahmed et al.\
\cite{Ahmed2026} showed that in feedforward networks of Stuart--Landau
oscillators, excitation-parameter inhomogeneity can genuinely enhance
signal amplification and broaden the phase-locking region.

Despite this progress, a systematic computational study comparing
synchronization regions for homogeneous and heterogeneous (barycentric-constrained)
parameter distributions across multiple canonical network topologies has
been lacking.  Three specific gaps exist in the literature:
\begin{enumerate}[label=(\roman*)]
\item The barycentric condition of \cite{Palacios2024} has not been
  systematically tested across diverse network topologies.
\item The connection between AISync \cite{Zhang2017} and
  normal-form disorder effects \cite{Palacios2024} has not been
  bridged experimentally.
\item No study has characterized which topological properties predict
  whether disorder helps or hurts synchronization.
\end{enumerate}

\subsection*{Summary of results}
We conduct systematic numerical experiments on Kuramoto and
Stuart--Landau oscillator models across six network topologies
(complete, ring, star, path, small-world, and circulant graphs).  Our
main findings are as follows.

\begin{enumerate}[label=(\arabic*)]
\item \textbf{Stuart--Landau feedforward networks}
  (Theorem~\ref{thm:SL-enhancement}): Excitation-parameter disorder
  nearly doubles the phase-locking region, with the phase-locking
  fraction increasing monotonically from $61\%$ to $100\%$ as disorder
  strength increases.  The direction of disorder along the feedforward
  chain determines whether it enhances or suppresses the network
  output.

\item \textbf{Kuramoto ring networks}
  (Theorem~\ref{thm:ring-enhancement}): For a fixed disorder
  realization, the order parameter can improve by up to $32\%$ near
  the synchronization transition.  However, averaging over random
  realizations eliminates the benefit
  (Proposition~\ref{prop:realization}), revealing that only specific
  disorder patterns aligned with the network topology enhance
  synchronization.

\item \textbf{Circulant graph survey}
  (Theorem~\ref{thm:circulant-fraction}): Between $20\%$ and $37\%$ of
  circulant graphs of order $6 \leq N \leq 10$ exhibit improved
  synchronization from disorder, with the simple ring $C_N(1)$
  (largest spectral gap ratio) benefiting most consistently.

\item \textbf{Necessity of structure}
  (Proposition~\ref{prop:barycentric-necessary}): The barycentric
  condition is necessary but not sufficient for disorder to enhance
  synchronization.  The spectral gap ratio of the Laplacian emerges as
  a predictor of which topologies can benefit.
\end{enumerate}

\subsection*{Proof techniques}
Our approach combines rigorous analysis of the linearized
synchronization dynamics with large-scale numerical experiments.  For
the Stuart--Landau feedforward network, we use a co-rotating frame
reduction following \cite{Ahmed2026} to obtain a two-parameter phase
diagram and prove monotonicity of the phase-locking fraction.  For
Kuramoto models, we analyze the Jacobian of the synchronized state and
relate disorder-induced changes in the effective Laplacian spectrum to
synchronization stability.  Statistical significance is assessed via
paired $t$-tests with $80$ trials per configuration.

\subsection*{Organization}
Section~\ref{sec:prelim} collects definitions and prerequisite results.
Section~\ref{sec:main} presents our main results with complete proofs.
Section~\ref{sec:discussion} discusses implications, connections to
related work, and open questions.  Section~\ref{sec:conclusion}
concludes.
