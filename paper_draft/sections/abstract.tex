We investigate the effect of zero-mean parameter heterogeneity on
synchronization in coupled oscillator networks.  Classical results
predict that heterogeneity raises the critical coupling threshold;
we show that this picture is incomplete.  Through systematic
numerical experiments on Kuramoto and Stuart--Landau oscillator models
across six network topologies, we establish three main findings.
First, for Stuart--Landau oscillators on feedforward networks,
excitation-parameter disorder satisfying the barycentric condition
nearly doubles the phase-locking region (from $35.6\%$ to $61.4\%$
of parameter space), with the phase-locking fraction increasing
monotonically to $100\%$ as disorder strength grows.  Second, for
Kuramoto oscillators on ring networks, specific zero-mean frequency
realizations improve the order parameter by up to $32\%$ near the
synchronization transition, but the enhancement is
realization-dependent: averaging over random realizations yields a
statistically nonsignificant effect.  Third, among circulant graphs of
order $6$, $8$, and $10$, between $20\%$ and $37\%$ exhibit improved
synchronization from disorder, with the simple ring (largest spectral
gap ratio) benefiting most consistently.  We formalize these
observations through propositions relating the spectral gap ratio of
the network Laplacian to the potential for disorder-enhanced
synchronization and prove that the barycentric condition is necessary
but not sufficient.  Our results demonstrate that beneficial disorder
is \emph{structured}: not any zero-mean heterogeneity helps, but
specific patterns aligned with the network topology do.
